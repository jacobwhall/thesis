% Options for packages loaded elsewhere
\PassOptionsToPackage{unicode}{hyperref}
\PassOptionsToPackage{hyphens}{url}
%
\documentclass[
]{article}
\usepackage{lmodern}
\usepackage{amssymb,amsmath}
\usepackage{ifxetex,ifluatex}
\ifnum 0\ifxetex 1\fi\ifluatex 1\fi=0 % if pdftex
  \usepackage[T1]{fontenc}
  \usepackage[utf8]{inputenc}
  \usepackage{textcomp} % provide euro and other symbols
\else % if luatex or xetex
  \usepackage{unicode-math}
  \defaultfontfeatures{Scale=MatchLowercase}
  \defaultfontfeatures[\rmfamily]{Ligatures=TeX,Scale=1}
\fi
% Use upquote if available, for straight quotes in verbatim environments
\IfFileExists{upquote.sty}{\usepackage{upquote}}{}
\IfFileExists{microtype.sty}{% use microtype if available
  \usepackage[]{microtype}
  \UseMicrotypeSet[protrusion]{basicmath} % disable protrusion for tt fonts
}{}
\makeatletter
\@ifundefined{KOMAClassName}{% if non-KOMA class
  \IfFileExists{parskip.sty}{%
    \usepackage{parskip}
  }{% else
    \setlength{\parindent}{0pt}
    \setlength{\parskip}{6pt plus 2pt minus 1pt}}
}{% if KOMA class
  \KOMAoptions{parskip=half}}
\makeatother
\usepackage{xcolor}
\IfFileExists{xurl.sty}{\usepackage{xurl}}{} % add URL line breaks if available
\IfFileExists{bookmark.sty}{\usepackage{bookmark}}{\usepackage{hyperref}}
\hypersetup{
  hidelinks,
  pdfcreator={LaTeX via pandoc}}
\urlstyle{same} % disable monospaced font for URLs
\setlength{\emergencystretch}{3em} % prevent overfull lines
\providecommand{\tightlist}{%
  \setlength{\itemsep}{0pt}\setlength{\parskip}{0pt}}
\setcounter{secnumdepth}{-\maxdimen} % remove section numbering

\author{}
\date{}

\begin{document}

\textbf{Senior Research Progress Report}\\
Jacob Hall\\
Geology Senior Research\\
Professor Heather Macdonald

\subsection{Project Title}

"Producing a Video to Communicate Radon Health Risks in Williamsburg"

\subsection{Description of Project Goals}

~~~~The goal of this project is to produce a video that conveys to
Williamsburg area residents information about radon. This will include
the formation of radon, its health risks to humans, how it forms in
Williamsburg, and how residents can detect and mitigate it. In this
report, I've divided the project into three parts:

\begin{enumerate}
\def\labelenumi{\arabic{enumi}.}
\item
  \textbf{Scriptwriting}\\Determining what will be said in the video.
\item
  \textbf{Production}\\The actual creation of the video, from audio
  recording to caption writing.
\item
  \textbf{Literature Review}\\The research and writing that backs my
  video as documentaton of the video's content and production.
\end{enumerate}

\subsection{Progress Thus Far}

~~~~The current plan is to embed into the story map
\href{https://storymaps.arcgis.com/stories/10f6d3d7c0014a1087fe3ef14f306520}{Radon
in Williamsburg Homes}. This will be a single site with all of the
resources Williamsburg residents need to learn about and address radon,
including my video, the risk map, and links to VDH's radon page as well
as other resources.

\subsubsection{Scriptwriting}

~~~~Over the summer, I developed the script with the supervision of my
advisors Professor Ibes, Heather, Jim, and Rick. In our meetings, my
advisors and I decided that the video should remain relatively short, in
order to maintain our audience's attention and to focus on the most
important pieces of information. Rick and I met a few times to discuss
the stratigraphy of Williamsburg, previous work at William \& Mary, and
his goals in researching radon. Rick helped me narrow the scope of my
video to the geologic concepts most relevant to this area, and showed me
radon concentration data collected in Williamsburg homes. I've been
meeting regularly with Professor Ibes to craft the script. Her guidance
on communicating the scientific concepts in the script has been key to
developing an approachable video without losing scientific accuracy. One
of the first things we determined were the sections of the script: Lede,
Health Concern, Background \& Research, Who Needs to Know, What Can You
Do, and Kicker. These section titles helped us make sure we addressed
each piece of the story we want to tell, and organize the script into a
building narrative. I have also met periodically with Heather and Jim to
review the script with regard to the overall goals of the project, and
for scientific accuracy of the content. The script is now a final draft,
and ready to be used in draft audio recordings as the video production
commences.

\subsubsection{Production}

~~~~Since the beginning of the semester, I've drafted graphics for the
video. I've discussed these visuals with each of my advisors, using
their feedback to develop each to better represent sections of the
video. This week, I made the first audio recording of the script, that I
will review with my advisors, before recording a final version.

\subsubsection{Literature Review}

~~~~In the scriptwriting process, I had to verify that the facts were backed
up by the research I had done, which meant updating the literature
review with new sources.

\subsection{What Remains}

\subsubsection{Production}

~~~~With the script nearing completion, it is coming time to focus on the
next pieces of the video production process. In the next month, I plan
to have finalized versions of the audio recording and graphics for the
video. With those pieces reviewed by my advisors, it will be time to
edit the video into a cohesive work. This is another process I hope to
be collaborative, with iterative drafts that I share with my advisors to
gauge its progress. My goal has been to have the final draft of the
video completed by the end of the semester, which will require multiple
editing sessions each week, and regular meetings to steer the direction
of my work.

\subsubsection{Literature Review}

~~~~The literature review for my project not only needs to provide the facts
used in the video, but demonstrate a broader understanding of radon, the
stratigraphy of Williamsburg, and science communication. This goal
represents what is left for my literature review: I intend to expand its
breadth give the reader a fuller picture than is conveyed in the video.
This is a process I'll seek my advisors' guidance on as the semester
progresses.

\subsection{Appendix}

\subsubsection{Current Script}

~~~~Did you know, several homes in the Williamsburg Area were found to have
hazardous levels of radon, and that radon is the second most common
cause of lung cancer in the United States? This odourless, invisible gas
does not cause immediate side-effects, so you may not know if you or
your family have been exposed. In this video, you'll learn what radon
is, why it is an urgent health concern, and how you can take action to
mitigate radon in your home.

~~~~Radon is a naturally-occurring radioactive gas that can enter your home
through multiple points including air ducts; cracks in your walls,
floors, or foundation; crawl spaces; construction joints; or your local
water supply. The highest levels of radon are usually found in the
lowest areas of a home. Radon levels can fluctuate with the seasons,
with the highest levels usually occurring during the winter months.

~~~~But, how did radon end up in Williamsburg? This explanation requires a
brief geology lesson. The ground beneath our feet consists of multiple
layers of sediment. One of these layers, called the Yorktown Formation,
is primarily composed of porous sand, shells, and marine mammal bones.
It turns out that fossils common in this formation, such as whale bones
and shark teeth, are a key source of uranium. Uranium naturally decays
into radon, which can then escape between the grains of sediment and
travel to the surface.

~~~~If your home is built on or near the Yorktown formation, there could be
radon seeping into your home, without you even knowing it.

~~~~And if you do have a radon problem, it's essential you take immediate
action to protect yourself and your family. Rest assured, there are
easy, affordable, and effective solutions. And you can start right now.
Simply follow these four steps:

~~~~Step 1: The William \& Mary Geology Department has an online,
interactive radon risk map of Williamsburg. Follow the map link in the
description to quickly determine your home's estimated level of risk.

~~~~Step 2: Obtain a radon test kit as soon as possible, especially if
your home is in a moderate to high risk area. Testing your home for
radon is easy, and takes only a few minutes. The Virginia Department of
Health provides a radon detection kit for just \$3, including shipping.
To order a kit, or to find professional testing and continuous
radon-monitoring systems, visit vdhradon.org.

~~~~Step 3: If radon is detected in your home, you will want to hire a
professional immediately. The National Radon Proficiency Program and
National Radon Safety Board both have directories of radon detection and
mitigation professionals. This information can be found at vdhradon.org
and in the video description. If you are renting, be sure to alert your
landlord to the problem, immediately. If you are a realtor, educate
yourself and your clients about radon risk and mitigation.

~~~~Step 4: Even if you do not detect radon in your home, conduct a
follow-up test each fall or winter, preferably during fair weather, when
radon levels are most accurate.

~~~~Now that you understand what radon is, why it can be dangerous, and what
you can do about it, be sure to share this video with friends, family,
and coworkers in Williamsburg. Together, we can mitigate the risks of
radon to assure the health of our community.

To learn more, check out our links in the video description.

\end{document}
